\clearpage
\sffamily
{\bfseries
\textcolor[rgb]{0.4,0.4,0.4}{Law 16 -- The Goal Kick} }
\phantomsection
\addcontentsline{toc}{subsection}{Law 16 -- The Goal Kick}


\bigskip

A goal kick is a method of restarting play.

\bigskip

A goal kick is awarded when the whole of the ball passes over the goal line, either on the ground or in the air, having last touched a player of the attacking team, and a goal is not scored in accordance with Law
10.

\bigskip

A goal may be scored directly from a goal kick, but only against the opposing team.
If the ball directly enters the kicker's goal a corner kick is awarded to the
opponents if the ball left the penalty area.

\bigskip

\bigskip

{\bfseries Procedure}

\headlinebox

If the ball leaves the field it will be replaced on the field by the referee%
\removed{ or an assistant referee}.
If the whole of the ball passes over the goal line the ball is placed on the
touch line at the intersection with the centre line on the side of the field
the ball went out.

\bigskip

Balls are deemed to be out based on the team that last touched the ball, irrespective of who actually kicked the ball. 

\bigskip

After placing the ball, the same procedure and rules of executing a direct free kick apply.

\bigskip

\greyed{
(replaces:

\begin{itemize}
\item The ball is kicked from any point within the goal area by a player of
the defending team
\item Opponents remain outside the penalty area until the ball is in play
\item The kicker must not play the ball again until it has touched another
player
\item The ball is in play when it is kicked directly out of the penalty area)
\end{itemize}
}

\simplify{
\bigskip

{\bfseries Infringements and sanctions}

\headlinebox

\greyed{
(suspended: If the ball is not kicked directly out of the penalty area
from a goal kick:

\begin{itemize}
\item the kick is retaken
\end{itemize}

\bigskip

Goal kick taken by a player other than the goalkeeper

If, after the ball is in play, the kicker touches the ball again (except
with his hands) before it has touched another player:

\begin{itemize}
\item an indirect free kick is awarded to the opposing team, to be taken from
the place where the infringement occurred (see Law 13 -- Position of
free kick)
\end{itemize}

\bigskip

If, after the ball is in play, the kicker deliberately handles the ball
before it has touched another player:

\begin{itemize}
\item a direct free kick is awarded to the opposing team, to be taken from the
place where the infringement occurred (see Law 13 -- Position of free
kick)
\item a penalty kick is awarded if the infringement occurred inside the
kicker's penalty area
\end{itemize}

\bigskip

Goal kick taken by the goalkeeper

If, after the ball is in play, the goalkeeper touches the ball again
(except with his hands) before it has touched another player:

\begin{itemize}
\item an indirect free kick is awarded to the opposing team, to be taken from
the place where the infringement occurred (see Law 13 -- Position of
free kick)
\end{itemize}

\bigskip

If, after the ball is in play, the goalkeeper deliberately handles the
ball before it has touched another player:

\begin{itemize}
\item a direct free kick is awarded to the opposing team if the infringement
occurred outside the goalkeeper's penalty area, to be
taken from the place where the infringement occurred (see Law 13 --
Position of free kick)
\item an indirect free kick is awarded to the opposing team if the
infringement occurred inside the goalkeeper's penalty
area, to be taken from the place where the infringement occurred (see
Law 13 -- Position of free kick)
\end{itemize}

\bigskip

In the event of any other infringement of this Law:

\begin{itemize}
\item the kick is retaken) 
\end{itemize}
}
}
